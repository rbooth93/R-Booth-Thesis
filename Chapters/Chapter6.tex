% Chapter Template

\chapter{Conclusions and Outlook} % Main chapter title

\label{Chapter6} % Change X to a consecutive number; for referencing this chapter elsewhere, use \ref{ChapterX}

%----------------------------------------------------------------------------------------
%	SECTION 1
%----------------------------------------------------------------------------------------

\epigraph{\itshape We keep moving forward, opening new doors and doing new things, because we're curious, and curiosity keeps leading us down new paths.}{---Walt Disney}

\section{Conclusions}

Stellar ages gives insight into the timescales of processes at work both within the star (such as the stellar dynamo) or in the environment around the star (e.g. exoplanet dynamics). However, ages are difficult to determine as they cannot be directly calculated from observations of stars. The most successful methods of determining stellar ages (which include stellar isochrones and asteroseismology) are still dependent on stellar models. One potential method of obtaining an estimate for the stellar age is to calibrate an age relationship based on an observable parameter. This is possible for cool stars with an outer convective zone as these stars lose angular momentum via magnetic braking while on the main sequence leading to a decrease in the rotational velocity of the star. This led to the concept of gyrochronology - the study of the rotation period as a function of age. The rotation period is inherently linked to the magnetic activity of the star through the stellar dynamo, therefore an activity-age relationship can also be constructed. With the advancement of asteroseismology allowing ages to be obtained for large samples of field stars, it is now possible to calibrate age relationships using asteroseismic ages. This is the focus of the work presented in this thesis, using stars with asteroseismic ages and observations for magnetic activity indicators to investigate the activity-age relationship. In this thesis I have presented work that has investigated the X-ray luminosity - age relationship, the chromospheric emission - age relationship and finally, the activity-rotation relationship. The results and conclusions for each of these investigations are summarised below.

\subsection{X-ray luminosity - age relationship}

In Chapter \ref{Chapter3}, I presented an investigation into the X-ray luminosity - age relationship for cool stars older than a gigayear. This study presented new X-ray observations of several cool stars along with analysis of archival data to form a sample of 24 stars. Ages for this sample of stars came predominantly from asteroseismology which provide more accurate ages than most other studies were able to provide. Using observations from the \Chandra and \XMM X-ray telescopes, X-ray luminosities were determined for fourteen stars primarily, with spectral modelling for eight of those stars with sufficient photon counts. For ten stars with non-significant detections, upper limits to their X-ray luminosity were calculated.

In order to perform an analysis on the full sample of stars, the mass bias that is present in the X-ray luminosity had to be taken into account. This was achieved by normalising the X-ray luminosity by stellar surface area. An orthogonal distance regression was then used to find the best-fitting relationship between the normalised X-ray luminosity and stellar age for cool stars older than a gigayear. An age-exponent value of $-2.80 \pm 0.72$ was found, representing a steepening of the relationship for cool stars older than a gigayear.

Possible explanations for this steepening of the activity - age relationship include the hypothesis that the rotational spin-down is more rapid than previously thought. However, a recent observational study by \citealt{van_Saders_etal_2016} investigated the rotational evolution of cool stars with asteroseismic ages and found that older stars seemed to be rotating more rapidly than expected. \citet{van_Saders_etal_2016} attribute weakened magnetic braking as the cause for the rapid rotation observed in these stars. The alternative explanation for the steepening of the activity-age relationship is the steepening of the rotation-activity relationship. There is some evidence for such a steepening as seen by \citet{Wright_etal_2011} by considering a small, unbiased sample of their sample. Considering the current evidence towards weakened magnetic braking, the results presented in Chapter \ref{Chapter3} would indicate a steepening of the rotation-activity relationship.

Whatever the underlying reason for the change in the activity - age relationship, our data indicates that there is such a steepening. Further studies incorporating age, activity and rotation will be crucial to understanding what components are responsible for the change seen.

\subsection{Chromospheric emission - age relationship}

In Chapter \ref{Chapter4}, I presented an investigation into the chromospheric emission - age relationship for cool stars older than a gigayear. The main investigation considered how the \Rprime activity indicator evolved over the main sequence lifetime of cool stars. In this part, 26 stars with asteroseismic ages along with calculated values for the \Rprime activity indicator were analysed in an attempt to calibrate the age-activity relationship. Relationships between the S index calculated in the \esp and \narval spectrographs and the \Smw index using stars from \cite{Duncan_etal_1991}. This allowed  for a conversion between to the \Smw and the original method by \citet{Noyes_etal_1984} for the calculation of the \Rprime indicator to be used. No strong correlation of the \Rprime indicator with age was found.

A comparison was made of the sample of old stars with previous age-activity relationships and found that for a given age our sample of stars tended to be more inactive than the relationship predicts, especially for the younger stars in the sample. A comparison was also made between the sample of old stars and the age-mass-metallicity-activity relationship from \citet{Lorenzo_Oliveira_etal_2016} to see if the inclusion of mass and/or metallicity could explain the dispersion of the sample considered in this work. Metallicity was found to have a greater effect on the shape of the age-activity relationship. The relationship between the measured value of the \Rprime indicator and the expected value from the \citet{Lorenzo_Oliveira_etal_2016} relationship was also investigated and it was found that the majority of the sample of old stars lie below the expected age-activity relationship for their mass and metallicity. Given that the majority of the sample follows this trend, this is consistent with the X-ray data presented in Chapter \ref{Chapter3} \citep{Booth_etal_2017}.

The sample of old field stars considered in this research also brings into question the suitability of the \Rprime activity indicator as an age indicator for stars older than a gigayear as previously seen in the literature. Due to the low correlation between age and activity for the sample of old field stars, caution would be advised when considering the \Rprime indicator as an age diagnostic for stars older than a gigayear.

An additional analysis into the \Halpha spectral line was also conducted as an alternative to the \caII lines. The \Halpha index was calculated for 26 stars and converted to the $I_{H\alpha}$ parameter \citep{Gomes_da_Silva_etal_2014}. However, upon further investigation into the \Halpha index and $I_{H\alpha}$ as a function of colour, the $I_{H\alpha}$ parameter was unsuccessful in removing the mass bias present in the sample. This is most likely due to the use of a conversion that is calibrated with spectra from a different spectrograph. In order to use the \Halpha spectral line as an activity indicator, a calibration to the \Halpha index calculated in the \textit{HARPS} spectrograph would be needed.

\subsection{Activity - rotation investigation}

\textcolor{red}{TBD - Once chapter 5 is finished!}

\section{Outlook}

%Currently just thinking of different things, will restructure once all finished
%asteroseismology
The field of asteroseismology is still fairly young, the number of solar-like oscillators detected was only significantly increased by space missions such as CoRoT, Kepler and K2. In order to determine precise ages for asteroseismic targets, detailed modelling of individual oscillation frequencies are required which has been achieved within the past few years \citet{Silva_Aguirre_etal_2017} INSERT REFS. However, the sample sizes with asteroseismic ages for age-activity-rotation relationships remain small. TESS, which is currently active, is expected to increase the number of stars with detected oscillations by an order of magnitude \citep{Schofield_etal_2019}, including exoplanet hosts \citep{Campante_etal_2016}. The TESS coverage of the sky is much larger than Kepler/K2 - it will survey over 85\% of the sky during the first two years of the mission. It will also observe stars that are much brighter than Kepler by several magnitudes; this will allow for asteroseismic analysis of near-by solar type stars that will have complementary data providing strong constraints for calculation of stellar ages.















