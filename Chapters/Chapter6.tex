% Chapter Template

\chapter{Conclusions and Outlook} % Main chapter title

\label{Chapter6} % Change X to a consecutive number; for referencing this chapter elsewhere, use \ref{ChapterX}

%----------------------------------------------------------------------------------------
%	SECTION 1
%----------------------------------------------------------------------------------------

\epigraph{\itshape We keep moving forward, opening new doors and doing new things, because we're curious, and curiosity keeps leading us down new paths.}{---Walt Disney}

\section{Conclusions}

Stellar ages gives insight into the timescales of processes at work both within the star (such as the stellar dynamo) or in the environment around the star (e.g. exoplanet dynamics). However, ages are difficult to determine as they cannot be directly calculated from observations of stars. The most successful methods of determining stellar ages (which include stellar isochrones and asteroseismology) are still dependent on stellar models. One potential method of obtaining an estimate for the stellar age is to calibrate an age relationship based on an observable parameter. This is possible for cool stars with an outer convective zone as these stars lose angular momentum via magnetic braking while on the main sequence leading to a decrease in the rotational velocity of the star. This led to the concept of gyrochronology - the study of the rotation period as a function of age. The rotation period is inherently linked to the magnetic activity of the star through the stellar dynamo, therefore an activity-age relationship can also be constructed. With the advancement of asteroseismology allowing ages to be obtained for large samples of field stars, it is now possible to calibrate age relationships using asteroseismic ages. This is the focus of the work presented in this thesis, using stars with asteroseismic ages and observations for magnetic activity indicators to investigate the activity-age relationship. In thesis I have presented work that has investigated the X-ray luminosity - age relationship, the chromospheric emission - age relationship and finally, the activity-rotation relationship. The results and conclusions for each of these investigations are summarised below.

\subsection{X-ray luminosity - age relationship}

In Chapter \ref{Chapter3}, I presented an investigation into the X-ray luminosity - age relationship for cool stars older than a gigayear. This study presented new X-ray observations of several cool stars along with analysis of archival data to form a sample of 24 stars. Ages for this sample of stars came predominantly from asteroseismology which provide more accurate ages than most other studies were able to provide. Using observations from the \Chandra and \XMM X-ray telescopes, X-ray luminosities were determined for fourteen stars primarily, with spectral modelling for eight of those stars with sufficient photon counts. For ten stars with non-significant detections, upper limits to their X-ray luminosity were calculated.

In order to perform an analysis on the full sample of stars, the mass bias that is present in the X-ray luminosity had to be taken into account. This was achieved by normalising the X-ray luminosity by stellar surface area. An orthogonal distance regression was then used to find the best-fitting relationship between the normalised X-ray luminosity and stellar age for cool stars older than a gigayear. An age-exponent value of $-2.80 \pm 0.72$ was found, representing a steepening of the relationship for cool stars older than a gigayear.

Possible explanations for this steepening of the activity - age relationship include the hypothesis that the rotational spin-down is more rapid than previously thought.

\subsection{Chromospheric emission - age relationship}

\subsection{Activity - rotation investigation}

\section{Outlook}


