% Chapter Template

\chapter{Activity - Rotation Investigation} % Main chapter title

\label{Chapter5} 

\epigraph{\itshape Sometimes the wheel turns slowly, but it turns.}{---Lorne Michaels}

\section{Introduction}

The results presented in Chapters \ref{Chapter3} and \ref{Chapter4} investigated the age-activity relationship through coronal and chromospheric emission. However, the age-activity relationship is a consequence of magnetic braking that removes angular momentum from the star on the main sequence. Thus, the age-activity relationship is inherently link to the rotational evolution. As discussed in Section \ref{Chp2_activity-rotation_lit_review}, this has led to studies on the activity-rotation relationship (see e.g. \citealt{Pizzolato_etal_2003,Wright_etal_2011}). The aim of this study was to collect literature values for the rotation period of the sample of stars considered in the two previous studies and investigate the activity-rotation relationship.

\subsection{Determining rotation periods}

There are several possible methods to determine the rotation period of a star; since the rotation periods considered in this work were collected from the literature and use various methods, they are summarised here for convenience.

The first method may be the most commonly used to determine the rotation period for a star - photometric variation in light curves due to starspots. As discussed in Section \ref{Chp1_starspots}, starspots are observed in light curves as periodic variations in the brightness of star as the starspot rotates in and out of view of the observer. To obtain the stellar rotation period, techniques are used to detect periodicities in the light curve. One such technique is the Lomb-Scargle periodogram \citep{Lomb_1976,Scargle_1982}, which uses a Fourier transform to search for periodicities. Typically, a distribution of peaks will appear in the power spectrum and the period with the largest Lomb-Scargle power will correspond to the rotation period. It is a commonly used technique to determine the stellar rotation period (see e.g. \citealt{do_Nascimento_etal_2014,Nielsen_etal_2013}).

Another technique used to determine periodic variations in light curve is the autocorrelation function (ACF); this method determines how correlated the light curve is with itself when offset by a certain time lag. This can be performed for a range of time lags with repeated spot crossings providing peaks in the ACF at time periods associated with the rotation period. An advantage of this method is that the shape of the periodicity in the light curve is not assumed and thus may be more useful in cases where the modulation is not perfectly sinusoidal. This technique is also fairly common in the literature and has been used for large samples of stars from \textit{Kepler} \citep{McQuillan_etal_2014}.

The second method used to determine stellar rotation periods is long term observations of magnetic activity indicators such as the chromospheric emission from the \caII or \Halpha spectral lines and even X-ray luminosity. It is known that these magnetic activity indicators trace active magnetic regions on the surface, thus if an active region crosses the stellar surface multiple times then the rotation period can be determined from the modulation in the magnetic activity indicator. This technique is ideal for stars with more subtle light curve modulation (e.g. slower rotators) as magnetic activity indicators can trace smaller active regions. Example of this method in the literature include \citet{Boro_Saikia_etal_2016,Robertson_etal_2015_GJ176,DeWarf_etal_2010}.

The third method that can be used to determine the rotation period is from asteroseismology. As discussed in Section \ref{Section:intro_ages}, asteroseismology is a valuable tool for determining fundamental parameters through observations of stellar oscillations. However, one aspect not discussed in the previous section is that any departure from solid body rotation will result in a frequency splitting of non-radial oscillation modes; thus the frequency will also depend on the azimuthal order, m. From helioseismology, the analysis of the frequency splitting due to rotation has provided detail about the interior of the Sun and revealed the presence of the tachocline which is instrumental in stellar dynamo theory \citep{Ossendrijver_2003}. However, in order to obtain 







































