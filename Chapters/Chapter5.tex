% Chapter Template

\chapter{Activity - Rotation Investigation} % Main chapter title

\label{Chapter5} 

\epigraph{\itshape Sometimes the wheel turns slowly, but it turns.}{---Lorne Michaels}

\section{Introduction}

The results presented in Chapters \ref{Chapter3} and \ref{Chapter4} investigated the age-activity relationship through coronal and chromospheric emission. However, the age-activity relationship is a consequence of magnetic braking that removes angular momentum from the star on the main sequence. Thus, the age-activity relationship is inherently link to the rotational evolution. As discussed in Section \ref{Chp2_activity-rotation_lit_review}, this has led to studies on the activity-rotation relationship (see e.g. \citealt{Pizzolato_etal_2003,Wright_etal_2011}). The aim of this study was to collect literature values for the rotation period of the sample of stars considered in the two previous studies and investigate the activity-rotation relationship.

\subsection{How to detect Rotation periods}








































