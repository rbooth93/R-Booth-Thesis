% Chapter Template

\chapter{Chromospheric emission - age relationship} % Main chapter title

\label{Chapter4} % Change X to a consecutive number; for referencing this chapter elsewhere, use \ref{ChapterX}

\epigraph{\itshape There is a way out of every box, a solution to every puzzle; it's just a matter of finding it.}{---Captain Jean-Luc Picard, \itshape Star Trek}

\section{Introduction}
\textcolor{red}{RB: If this paper is published before submitting - put paragraph similar to the one in Chapter 3 here!}

In \citet{Booth_etal_2017} a steepening of the age-activity relationship was found using X-ray observations coupled to asteroseismic ages. While this study provided the first step to studying the X-ray luminosity-age relationship beyond a gigayear, using the X-ray luminosity as a magnetic activity indicator may not be the most accessible indicator as it requires significant observational resources on an X-ray telescope. An alternative magnetic activity indicator is chromospheric emission in spectral lines such as \caII or H$-\alpha$. Particularly for exoplanet host stars, spectra are usually obtained to confirm the presence of candidate exoplanets through measurement of radial velocity. 









ocities

 vel





