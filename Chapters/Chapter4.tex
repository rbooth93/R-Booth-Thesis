% Chapter Template

\chapter{Chromospheric emission - age relationship} % Main chapter title

\label{Chapter4} % Change X to a consecutive number; for referencing this chapter elsewhere, use \ref{ChapterX}

\epigraph{\itshape There is a way out of every box, a solution to every puzzle; it's just a matter of finding it.}{---Captain Jean-Luc Picard, \itshape Star Trek}

\section{Introduction}
\textcolor{red}{If this paper is published before submitting - put paragraph similar to the one in Chapter 3 here!}

The chromospheric emission - age relationship has a long history dating back to \citet{Wilson_1963} who first postulated that the chromospheric emission seen in \caII (and magnetic activity in general) declines with stellar age. It was first quantitatively seen in \citet{Skumanich_1972} that showed that the calcium emission declined with age according to the relationship: $Ca^{+} \propto \tau^{\frac{1}{2}}$. However, in the forty years that have passed,

